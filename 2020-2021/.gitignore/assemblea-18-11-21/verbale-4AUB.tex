% ============================ Enrico Ribiani 16-03-2021 ====================================================================
% -------------classe documento -------------------
\documentclass{article}
% ------------ pacchetti necessari ----------------

\usepackage{graphicx}   % need for figure
\usepackage{biblatex}   % citazioni
\usepackage{amsfonts}   % if you want the fonts
\usepackage{amssymb}    % if you want extra symbols
\usepackage{graphicx}   % need for figures
\usepackage{mathptmx}
\usepackage{float}% ---------====== MOLTO IMPORTANTE PER METTERE TABELLE E FIGURE DOVE SI VUOLE ES \begin{coso}[H]
\usepackage[utf8]{inputenc}
\usepackage{textcomp}
\usepackage[hang,flushmargin,bottom]{footmisc} % footnote format
\usepackage{fancyhdr, lastpage}
\usepackage{titlesec}
\titleformat{\section}{\normalsize\bfseries}{\thesection.}{1em}{}	% required for heading numbering style
%===================inizio pagina del titolo=================
\begin{document}
\begin{center}
    \begin{figure}
        \includegraphics{~/varie/logo.png}
    \end{figure}
\end{center}
\hline\\

\vspace{1cm}
\Huge Verbale Assemblea di Classe 4AUB\\
\vspace{0.5cm}
\begin{flushleft}
    \normalsize
    Il giorno 18 del mese di novembre 2021, si è riunita l'assemblea di classe 4AUB per discutere il seguente 
    ordine del giorno:
    \begin{itemize}
        \item Gestione cellulari
        \item Programmazione prove scritte
        \item Programmazione relazioni laboratoriali
        \item Attività educazione civica alla cittadinanza
        \item Chiarimento problemi con docenti 
        \item Varie ed eventuali
    \end{itemize}
\vspace{1cm}
L'assemblea si è riunita alle ore 10:25 e si è sciolta alle ore 11:15.\\
Il segretario verbalista eletto per questa assemblea è stato Marco Ciola.\\
Entrambi i rappresentanti Enrico Ribiani e Simone Lunelli erano presenti.\\
\vspace{0.5 cm}
La classe si è confrontata sugli argomenti dell'o.d.g. partendo dal primo punto:
\begin{center}
    \large \textbf{Gestione cellulari}
\end{center}
L'assemblea si è trovata d'accordo che l'utilizzo improprio dello smartphone perpetrato da buona parte della
classe soprattutto in determinate ore rappresenti un problema soprattutto quando viene reiterato in seguito 
a un richiamo da parte del docente.\\
La classe si è detta all'unanimità di essere collaborativa per risolvere questo problema.\\
\vspace{0.2 cm}

\begin{center}
    \large \textbf{Programmazione prove scritte}
\end{center}
Dopo aver notato un sovraccarico nel primo periodo dato dalla mole di prove scritte la classe si è dichiarata
soddisfatta dall'attuale programmazione e coordinazione tra docenti infatti la calendarizzazione delle prove 
scritte è stata eseguita in modo ottimale, con largo anticipo e distacco tra una verifica e l'altra.\\
\vspace{0.2 cm}

\begin{center}
    \large \textbf{Programmazione relazioni laboratoriali}
\end{center}
Si richiede flessibilità sulla data di consegna delle relazioni nel caso in cui nella stessa settimana ci fossero in programma molteplici
prove scritte e relazioni di diverse materie.\\

\vspace{0.2 cm}

\begin{center}
    \large \textbf{Attività educazione civica alla cittadinanza}
\end{center}
L'assemblea ha chiesto con unanimità se fosse possibile differenziare le attività di ECC dalle ripetitive ricereche/relazioni 
in modo da ottenere una maggiore attenzione e coinvolgimento da parte della classe.\\
La classe si è detta disponibile a collaborare con i docenti perché conscia delle difficoltà intrinseche che questa nuova materia
comporta.\\
\vspace{0.2 cm}

\begin{center}
    \large \textbf{Chiarimento problemi con docenti }
\end{center}
La maggioranza dell'assemblea ha chiesto maggiore chiarezza nelle correzioni/valutazioni delle domande aperte nelle prove scritte
e si è detta piu propensa per una modalità di verifica mista tra domande aperte ed esercizi alternativi.\\
La classe è soddisfatta con la gestione delle interrogazioni.\\

\vspace{0.2 cm}

\begin{center}
    \large \textbf{Varie ed eventuali}
\end{center}
L'assemblea ha discusso sul fatto che sebbene sia ancora relativamente presto verrebbe molto apprezzata più informazione sull'alternanza
scuola-lavoro che la scuola offre, sia per il periodo estivo che duranmte l'anno scolastico.\\
\\
Il secondo punto emerso nella parte finale dell'assemblea sono stati i viaggi di istruzione e le attività, la classe ha molto 
apprezzato l'uscita svolta all'AcroPark di Malè e si è dichiarata volenterosa di svolgerne altre, soprattutto riguardant ambiti scolastici.\\
Durante la discussione di questo argomento è sono state tenute presenti tutte le problemantiche dovute alla pandemia e le restrizioni 
da essa conseguenti.\\
\end{flushleft}
\vspace{1cm}
\normalsize
Il Rappresentante \hspace{3cm} Il Rappresentante\\
Enrico Ribiani \hspace{3.4cm} Simone Lunelli\\

\end{document}