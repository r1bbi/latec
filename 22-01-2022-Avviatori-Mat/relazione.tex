% ============================ Enrico Ribiani 16-03-2021 ====================================================================
% Base per i documenti  
\documentclass[12pt]{article}
% ------------ pacchetti necessari ----------------
\usepackage[a4paper, total={6in, 8in},margin=1in]{geometry} % formattazione decente della pagina
\usepackage{graphicx}                            % need for figure
\usepackage{amsmath}
\usepackage{amsfonts}                            % if you want the fonts
\usepackage{amssymb}                             % if you want extra symbols
\usepackage{graphicx}  
\renewcommand{\figurename}{Figura}  
\renewcommand{\contentsname}{Indice}                        % need for figures
\usepackage{mathptmx}
\usepackage{float}                               % serve per mettere tabelle e immagini dove si vuole 
\usepackage[utf8]{inputenc}
\usepackage{textcomp}
\usepackage[hang,flushmargin,bottom]{footmisc}   % footnote format
\usepackage{fancyhdr, lastpage}
\usepackage{titlesec}
\usepackage[table,dvipsnames]{xcolor}
%\pagestyle{fancy}
%\renewcommand{\headrulewidth}{0pt}
%\renewcommand*\contentsname{Indice}
\titleformat{\section}{\normalsize\bfseries}{\thesection.}{1em}{}	% required for heading numbering style
\titleformat*{\section}{\Large\bfseries}
\titleformat*{\subsection}{\large\bfseries}
%\usepackage{siunitx}
%\usepackage{tikz}
\usepackage{circuitikz}
%\usepackage[siunitx]{circuitikz}
\usepackage{multirow}
\usepackage{tikz}
\usepackage{amsmath}
\usetikzlibrary{angles,quotes}
\usepackage{placeins}

\usepackage{wasysym}
%===================links=================
\usepackage{hyperref}
\hypersetup{
    colorlinks=true,
    linkcolor=Sepia,
    filecolor=Green,      
    urlcolor=Cyan,
    pdftitle={SAMPLE},
    pdfpagemode=FullScreen,
    }
%===================inizio pagina del titolo=================
\begin{document}
    \begin{titlepage}
    \begin{center}
% ------------------ inizio immagine logo ----------
\begin{figure}
    \centering
    \includegraphics{~/varie/logo.png}
    \label{fig:logo}
\end{figure}
% ------------------ fine immagine logo ----------
% ------------------ fine immagine logo ----------
-------------------------------------------------------------------------------------\\
\vspace{2\baselineskip}
\large Enrico Ribiani\\
\large 4AUB\\
\vfill

\Huge{\textbf{Avviatore M.A.T}}\\
\vfill

\LARGE{Relazione n°3}\\
\vfill
\large{22-01-2022}
\end{center}
%=============== fine pagina titolo ===============
\end{titlepage}
\tableofcontents
\vskip 1cm
\section{Introduzione}
%Durante le ore di laboratorio di TPSE ogni studente è stato incaricato di disegnare e montare un quadro elettrico in grado di far avviare e fermare correttamente un motore asincrono trifase avendo a disposizione il software di disegno \textit{SPAC Automazione} e avendo i componenti elencati nella sezione ~\ref{Componenti}.\\
%Sia che il disegno che il montaggio e il collaudo sono stati eseguiti in 4 sessioni di laboratorio da due ore e 30 l'una per poi concludere scrivendo una relazione tecnica.
Questa relazione tecnica riguarda l'esercitazione laboratoriale riguardante gli avviatori di motori asincroni trifase svolte in  4 lezioni di laboratorio da due ore e 30 cadauna.\\
La classe è stata incaricata di svolgere il disegno utilizzando il software di disegno \textit{SPAC Automazione} e il montaggio dell'avviatore collegato a un motore senza carico per scopo didattico.\\
\section{Funzionamento}
Il circuito era molto semplice in quanto era la prima volta che la classe svolgeva un esperieza simile, infatti l'utilizzo era limitato a far partire il motore e a farlo fermare.\\ 
Il quadro era dotato di tre spie, una verde che si illuminava quando il motore era alimentato, una gialla accesa quando il relè termico entrava in funzione, e una rossa quando il motore era in moto.\\
%Per comandare il motore erano necessari infatti solamente due pulsanti, uno di start(normalmente aperto) e uno di stop(normalmente chiuso), ovviamente uno serviva a far partire il motore e l'altro a fermarlo quando in funzione facendo aprire i contatti che lo collegavano al motore disattivando la bobina ausiliaria.\\
Il pulsante di start serve ad alimentare la bobina che va a connettere il motore alle tre fasi chiudendo i contatti principali normalmente aperti, la bobina viene mantenuta eccitata tramite il parallelo.\\ 
Al contrario invece il pulsante di stop se premuto va a sconnettere la bobina dall'alimentazione che farà lo stesso con il motore, un altra causa per il quale il motore una volta azionato si spegne è l'attivazione del relè termico che spegne la bobina facendo aprire i contatti.\\
Il motore era dotato di tutte le protezioni già montate come fusibili, relè termici, sezionatori.\\
Il circuito avviatore ha il compito di mettere in moto il motore e di far passare per un determinato lasso temporale le sovracorrenti necessarie all'avviamento, per questo motivo il motore necessita di varie protezioni previste dalla norma \textit{CEI EN 60 947}.\\
Si può notare come all'interno dello stesso circuito ci sia una divisione tra parte di potenza e parte di comando.\\
La parte di potenza si differisce da quella di comando perché alimentata dalla tensione di rete trifase a 400V formata dal motore, le sue protezioni e il trasformatore che alimenta con 24V bifase la parte di comando.\\
Questa parte del circuito ha anche il compito di interfacciare l'operatore al motore tramite i due pulsanti sopracitati e le tre spie.\\
\section{Componenti}
\label{Componenti}
\begin{itemize}
    \item \emph{Q1} contattore
          
    \item \emph{Q2} sezionatore
          
    \item \emph{P1} spia gialla 
    \item \emph{P2} spia rossa
    \item \emph{P3} spia verde 
    \item \emph{S1} pulsante di start
    \item \emph{S0} pulsante di stop
    \item \emph{F1,F2,F3,F5} fusibili
    \item \emph{F4} relè termico
    \item \emph{T1} trasformatore
\end{itemize}
\vspace{15pt}
\begin{Large}
\textbf{Protezioni} 
\end{Large}
\subsection*{Protezione dalle sovracorrenti}
Il motore viene protetto dalle sovracorrenti che vanno a danneggiare il motore tramite il correlato aumento delle temperature, a questo serve il relè termico che in caso di sovraccorrenti e dalla durata dell'esposizione del motore ad esse fa staccare l'alimentazione alla bobina scollegando il motore e impedendone il danneggiamento.\\
Non deve agire durante l'avviamento.
\subsection*{Sezionatori}
Vengono utilizzati per scollegare a monte il motore in caso di manutenzione.\\
Questo componente non deve interrompere correnti maggiori di quelle nominali e serve solamente per scollegare il motore dalla rete elettrica.\\
\subsection*{Contattore}
Serve a stabilire o arrestare la corrente tramite i contatti principali e ad interfacciarsi con il circuito di comanddo tramite i suoi contatti ausiliari. Per questo motivo viene considerato un dispositivo di manovra.\\
Per il dimensionamento i parametri principali considerati sono stati il potere di interruzione in particolare con le sovracorrenti tipiche dell'avviamento che deve sopportare per periodi stabiliti di tempo e se richiesto deve interromperle.
\subsection*{Fusibili}
I fusibili servono a interrompere tempestivamente le correnti di cortocircuito impedendo la rottura di componenti del motore.
\section{Materiali}
\begin{itemize}
    \item Filo da \diameter1,5 cm  di colore rosa
    \item Filo da \diameter1 cm  di colore grigio
\end{itemize}
\section{Allegati:}
\newpage


\end{document}