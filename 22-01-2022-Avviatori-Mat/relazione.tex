% ============================ Enrico Ribiani 16-03-2021 ====================================================================
% Base per i documenti  
\documentclass[12pt]{article}
% ------------ pacchetti necessari ----------------
\usepackage[a4paper, total={6in, 8in},margin=1in]{geometry} % formattazione decente della pagina
\usepackage{graphicx}                            % need for figure
\usepackage{amsmath}
\usepackage{amsfonts}                            % if you want the fonts
\usepackage{amssymb}                             % if you want extra symbols
\usepackage{graphicx}  
\renewcommand{\figurename}{Figura}  
\renewcommand{\contentsname}{Indice}                        % need for figures
\usepackage{mathptmx}
\usepackage{float}                               % serve per mettere tabelle e immagini dove si vuole 
\usepackage[utf8]{inputenc}
\usepackage{textcomp}
\usepackage[hang,flushmargin,bottom]{footmisc}   % footnote format
\usepackage{fancyhdr, lastpage}
\usepackage{titlesec}
\usepackage[table,dvipsnames]{xcolor}
%\pagestyle{fancy}
%\renewcommand{\headrulewidth}{0pt}
%\renewcommand*\contentsname{Indice}
\titleformat{\section}{\normalsize\bfseries}{\thesection.}{1em}{}	% required for heading numbering style
\titleformat*{\section}{\Large\bfseries}
\titleformat*{\subsection}{\large\bfseries}
%\usepackage{siunitx}
%\usepackage{tikz}
\usepackage{circuitikz}
%\usepackage[siunitx]{circuitikz}
\usepackage{multirow}
\usepackage{tikz}
\usepackage{amsmath}
\usetikzlibrary{angles,quotes}
\usepackage{placeins}

\usepackage{wasysym}
%===================links=================
\usepackage{hyperref}
\hypersetup{
    colorlinks=true,
    linkcolor=Sepia,
    filecolor=Green,      
    urlcolor=Cyan,
    pdftitle={SAMPLE},
    pdfpagemode=FullScreen,
    }
%===================inizio pagina del titolo=================
\begin{document}
    \begin{titlepage}
    \begin{center}
% ------------------ inizio immagine logo ----------
\begin{figure}
    \centering
    \includegraphics{~/varie/logo.png}
    \label{fig:logo}
\end{figure}
% ------------------ fine immagine logo ----------
% ------------------ fine immagine logo ----------
-------------------------------------------------------------------------------------\\
\vspace{2\baselineskip}
\large Enrico Ribiani\\
\large 4AUB\\
\vfill

\Huge{\textbf{Avviatore M.A.T}}\\
\vfill

\LARGE{Relazione n°2}\\
\vfill
\large{22-01-2022}
\end{center}
%=============== fine pagina titolo ===============
\end{titlepage}
\tableofcontents
\vskip 1cm
\section{Introduzione}
Durante le ore di laboratorio di TPSE ogni studente è stato incaricato di disegnare e montare un quadro elettrico in grado di far avviare e fermare correttamente un motore asincrono trifase avendo a disposizione il software di disegno \textit{SPAC Automazione} e avendo i componenti elencati nella sezione ~\ref{Componenti}.\\
Sia che il disegno che il montaggio e il collaudo sono stati eseguiti in 4 sessioni di laboratorio da due ore e 30 l'una per poi concludere scrivendo una relazione tecnica.
\section{Funzionamento}
Il circuito era molto semplice in quanto era la prima volta che la classe svolgeva un esperieza simile, infatti l'utilizzo era limitato a far partire il motore e a farlo fermare, il quadro era dotato di tre spie, una verde che si illuminava quando il motore era alimentato, una gialla accesa quando il relè termico entrava in funzione, e una rossa quando il motore era in funzione.\\
%Per comandare il motore erano necessari infatti solamente due pulsanti, uno di start(normalmente aperto) e uno di stop(normalmente chiuso), ovviamente uno serviva a far partire il motore e l'altro a fermarlo quando in funzione facendo aprire i contatti che lo collegavano al motore disattivando la bobina ausiliaria.\\
Il pulsante di start serve ad alimentare la bobina che va a chiudere tutti i contatti andando a connettere il motore alle tre fasi, al contrario invece il pulsante di stop se premuto va a sconnettere la bobina dall'alimentazione che farà lo stesso con il motore, un altra causa per il quale il motore una volta azionato si spegne è l'attivazione del relè termico che spegne la bobina facendo aprire i contatti.
Il motore era dotato di tutte le protezioni già montate come fusibili, relè termici, sezionatori.\\
Il circuito avviatore ha il compito di mettere in moto il motore e di far passare per un determinato lasso temporale le sovracorrenti necessarie all'avviamento, per questo motivo il motore necessita di varie protezioni previste dalla norma \textit{CEI EN 60 947}.\\


\section{Componenti}
\label{Componenti}
\begin{itemize}
    \item Contattore %[Groda7472347]
    \item Pulsante/spia gialla 
    \item Pulsante/spia rosso
    \item Pulsante/spia verde 
    \item Morsettiere da X
    \item Morsettiera spie da Y
\end{itemize}
\section{Materiali}
\begin{itemize}
    \item Filo da \diameter1,5mm  di colore rosa
    \item Filo da \diameter1 mm  di colore grigio
\end{itemize}
\section{Allegati}
\newpage


\end{document}