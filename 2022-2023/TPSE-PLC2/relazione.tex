% ============================ Enrico Ribiani 16-03-2021 ====================================================================
% Base per i documenti  
\documentclass[12pt]{article}
% ------------ pacchetti necessari ----------------
\usepackage[a4paper, total={6in, 8in},margin=1in]{geometry} % formattazione decente della pagina
\usepackage{graphicx}                            % need for figure
\usepackage{amsmath}
\usepackage{amsfonts}                            % if you want the fonts
\usepackage{amssymb}                             % if you want extra symbols
\usepackage{graphicx}  
\renewcommand{\figurename}{Figura}  
\renewcommand{\contentsname}{Indice}                        % need for figures
\usepackage{mathptmx}
\usepackage{float}                               % serve per mettere tabelle e immagini dove si vuole 
\usepackage[utf8]{inputenc}
\usepackage{textcomp}
\usepackage[hang,flushmargin,bottom]{footmisc}   % footnote format
\usepackage{fancyhdr, lastpage}
\usepackage{titlesec}
\usepackage[table,dvipsnames]{xcolor}
%\pagestyle{fancy}
%\renewcommand{\headrulewidth}{0pt}
%\renewcommand*\contentsname{Indice}
\titleformat{\section}{\normalsize\bfseries}{\thesection.}{1em}{}	% required for heading numbering style
\titleformat*{\section}{\Large\bfseries}
\titleformat*{\subsection}{\large\bfseries}
%\usepackage{siunitx}
%\usepackage{tikz}
\usepackage{circuitikz}
%\usepackage[siunitx]{circuitikz}
\usepackage{multirow}
\usepackage{tikz}
\usepackage{amsmath}
\usepackage{shorttoc}
\usetikzlibrary{angles,quotes}
\usepackage{placeins}

\usepackage{wasysym}
%===================links=================
\usepackage{hyperref}
\hypersetup{
    colorlinks=true,
    linkcolor=darkgray,
    filecolor=Green,      
    urlcolor=Cyan,
    pdftitle={SAMPLE},
    pdfpagemode=FullScreen,
    }
%===================inizio pagina del titolo=================
\begin{document}
    \begin{titlepage}
    \begin{center}
% ------------------ inizio immagine logo ----------
\begin{figure}
    \centering
    \includegraphics{~/varie/logo.png}
    \label{fig:logo}
\end{figure}
% ------------------ fine immagine logo ----------
% ------------------ fine immagine logo ----------
-------------------------------------------------------------------------------------\\
\vspace{2\baselineskip}
\large Enrico Ribiani\\
\large Nicolò Ozretic\\

\large 5AUB\\
\vfill

\Huge{\textbf{Nastro piano sollevatore comandato da PLC}}\\
\vfill

\LARGE{Relazione n°1}\\
\vfill
\large{14-10-2022}
\end{center}
%=============== fine pagina titolo ===============
\end{titlepage}
\tableofcontents
\vskip 1cm
\section{Introduzione}
Questa relazione tecnica riguarda il sistema di controllo tramite PLC di una macchina composta una rulliera
da due nastri trasportatori e un piano alzatore.\\
Il sistema deve funzionare in modo automatico.\\
Sono allegati il disegno esplicativo, gli schemi del plc e il programma ladder con cui verrà programmato
il PLC.\\


\section{Funzionamento}
Il materiale trasportato arriva sul primo nastro trasportatore tramite una rulliera, se il pulsante di start 
viene premuto il nastro fa spostare il materiale fino al primo finecorsa.\\
Questo fine corsa fa alzare il piano elevatore che si ferma raggiunto il secondo finecorsa ossia all'altezza 
del secondo nastro, quindi entrambi i nastri entrano in funzione fino al terzo finecorsa che è posto al punto 
di prelevamento manuale del materiale.\\
Attivato il terzo finecorsa il piano alzatore si abbassa al punto di partenza e il ciclo riparte automaticamente
per altre 4 volte, dopodichè si ferma e per farlo ripartire ci sarà bisogno di premere nuovamente il pulsante
di start.\\
Come da normativa il sistema è dotato di un pulsante di emergenza per far fermare immediatamente i componenti
in movimento e delle lampade che di segnalazione che indicano quando i macchinari sono in azione, quando il
sistema è alimentato e quando è bloccato dal pulsante di emergenza.\\
\section{Componenti}

\begin{itemize}
    \item \textbf{Parte di potenza}
\end{itemize}
\noindent
bla
bla
bla
\begin{itemize}
    \item \textbf{Parte di comando}
\end{itemize}
bla
bla
bla
\section{Manuale d'uso????}


\section{Allegati}
\subsection{Norme di riferimento}
\subsection{Programma Ladder}
\subsection{Disegno esplicaivo}

\newpage


\end{document}