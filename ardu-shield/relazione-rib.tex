% ============================ Enrico Ribiani 16-03-2021 ====================================================================
% Base per i documenti  
\documentclass{article}
% ------------ pacchetti necessari ----------------
\usepackage[a4paper, total={6in, 8in}]{geometry} % formattazione decente della pagina
\usepackage{graphicx}                            % need for figure
\usepackage{amsmath}
\usepackage{amsfonts}                            % if you want the fonts
\usepackage{amssymb}                             % if you want extra symbols
\usepackage{graphicx}                            % need for figures
\usepackage{mathptmx}
\usepackage{float}                               % serve per mettere tabelle e immagini dove si vuole 
\usepackage[utf8]{inputenc}
\usepackage{textcomp}
\usepackage[hang,flushmargin,bottom]{footmisc}   % footnote format
\usepackage{fancyhdr, lastpage}
\usepackage{titlesec}
\usepackage[table,dvipsnames]{xcolor}

\titleformat{\section}{\normalsize\bfseries}{\thesection.}{1em}{}	% required for heading numbering style
\titleformat*{\section}{\Large\bfseries}
\titleformat*{\subsection}{\large\bfseries}
%===================links=================
\usepackage{hyperref}
\hypersetup{
    colorlinks=true,
    linkcolor=Sepia,
    filecolor=magenta,      
    urlcolor=blue,
    pdftitle={Arduino Shield},
    pdfpagemode=FullScreen,
    }
%===================inizio pagina del titolo=================
\begin{document}
    \begin{titlepage}
\begin{flushleft}
\vspace{3\baselineskip}

\Huge{\textbf{Progetto scheda di interfaccia per Arduino}}
\vfill
\LARGE Enrico Ribiani\\
\LARGE 3AUB\\
\vfill
\huge{ITT M. BUONARROTI 24-05-2021}

%=============== fine pagina titolo ===============
\end{flushleft}
\end{titlepage}
%=============== Intestazione ===============
\pagestyle{fancy}
\fancyhead{}
\fancyhead[RO,LE]{Enrico Ribiani}
\fancyhead[LO,RE]{3AUB}

\fancyhead[CO,CE]{sezione \thesection}%??????????????''
%=============== fine Intestazione ===============

\tableofcontents
\vskip 3cm

\section{Uso della scheda e obiettivi}
\section{Strumenti utilizzati}
\section{Progettazione, selezione componenti e dimensionamento}
    \subsection{Preventivo}
        \begin{center}
        \begin{tabular}{| c | c | c | c| c |} 
        \hline
        \rowcolor{BurntOrange} Codice prodotto & Descrizione & Quantità & Prezzo & Totale\\ [0.5ex] 
        \hline
        \rowcolor{Peach} 13dwer & groda & 8 & 8.88 & 64\\
        \hline
        \rowcolor{Apricot} 13dwer & groda & 8 & 8.88 & 64\\
        \hline
        \rowcolor{Peach} 13dwer & groda & 8 & 8.88 & 64\\
        \hline
        \rowcolor{Apricot}13dwer & groda & 8 & 8.88 & 64\\
        \hline
        \rowcolor{Peach} 13dwer & groda & 8 & 8.88 & 64\\
        \hline
        \rowcolor{Apricot} 13dwer & groda & 8 & 8.88 & 64\\
        \hline
   \end{tabular}
   \\
   \vskip 2mm
   \begin{tabular}[h]{|c|c|}
       \hline
        \rowcolor{BurntOrange} Prezzo totale & 121321\\
       \hline
   \end{tabular}
   \end{center}

\section{allegati}
\paragraph{Schema di multisim}
\href{https://drive.google.com/file/d/1VOPnspiu-4T2ZOR6uaUWNH1DUEd0fUcg/view?usp=sharing}{Link immagine}
\begin{figure}[H]
    \centering
        \includegraphics[scale=0.5]{/home/rib/varie/multisimp.png}
        \label{multisimp}
\end{figure}

\paragraph{Schema di Ultiboard}
\href{https://drive.google.com/file/d/1ZlJ_AIXvgzdlvAawX5noBu48i03zm5dP/view?usp=sharing}{Link immagine}
\begin{figure}[H]
    \centering
        \includegraphics[scale=0.5]{/home/rib/varie/ultiabroad.png}
        \label{ultiabroad}
\end{figure}

\paragraph{Render 3d}
\href{https://drive.google.com/file/d/1EMIzjUSU50ij58dLTjCctJGQM4YIgIii/view?usp=sharing}{Link immagine}
\begin{figure}[H]
    \centering
        \includegraphics[scale=0.5]{/home/rib/varie/3dboard.png}
        \label{3d}
\end{figure}
\end{document}
