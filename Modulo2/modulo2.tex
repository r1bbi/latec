% ============================ Enrico Ribiani 16-03-2021 ====================================================================
% Base per i documenti  
\documentclass{article}
% ------------ pacchetti necessari ----------------
\usepackage[a4paper, total={6in, 8in}]{geometry} % formattazione decente della pagina
\usepackage{graphicx}                            % need for figure
\usepackage{amsmath}
\usepackage{amsfonts}                            % if you want the fonts
\usepackage{amssymb}                             % if you want extra symbols
\usepackage{graphicx}                            % need for figures
\usepackage{mathptmx}
\usepackage{float}                               % serve per mettere tabelle e immagini dove si vuole 
\usepackage[utf8]{inputenc}
\usepackage{textcomp}
\usepackage[hang,flushmargin,bottom]{footmisc}   % footnote format
\usepackage{fancyhdr, lastpage}
\usepackage{titlesec}
\titleformat{\section}{\normalsize\bfseries}{\thesection.}{1em}{}	% required for heading numbering style

\titleformat*{\section}{\Large\bfseries}
\titleformat*{\subsection}{\large\bfseries}
%\titleformat*{\subsubsection}{\large\bfseries}
%===================inizio pagina del titolo=================
\begin{document}
    \begin{titlepage}
    \begin{center}
% ------------------ inizio immagine logo ----------
\begin{figure}
    \centering
    \includegraphics{/home/rib/varie/logo.png}
    \label{fig:logo}
\end{figure}
% ------------------ fine immagine logo ----------
% ------------------ fine immagine logo ----------
-------------------------------------------------------------------------------------\\
\vspace{2\baselineskip}
\large Enrico Ribiani\\
\large 3AUB\\
\vfill

\Huge{\textbf{Riassunto Modulo 2, Unità 2-3-4}}\\
\vfill

%\LARGE{numero esperienza}\\
%\vfill
\LARGE{\textbf{19-04-2021}}\\
%\vfill
\end{center}
%=============== fine pagina titolo ===============
\end{titlepage}
\tableofcontents
\vspace{3\baselineskip}

\section{Unità 2: Resistori}

\subsection{Parametri e caratteristiche}
I resistori sono uno dei componenti passivi più frequenti all'interno di un circuito\\
Essi comportano l'introduzione di una \textbf{resistenza} elettrica nel circuito, misurata in ohm [$\Omega$].
I principali parametri ossia quelli più importanti, che vanno a influire maggiormente sulla scelta di un resistore sono:\\
\begin{enumerate}
    \item valore nominale
    \item tolleranza
    \item potenza nominale
\end{enumerate}
Il \textbf{valore nominale} rappresenta il valore più probabile a cui fare riferimento durante il progetto del circuito.\\
La \textbf{tolleranza} rappresenta il campo di valori entro cui il valore reale della resistenza è compreso, spesso si esprime in maniera relativa al valore nominale.\\
La \textbf{potenza nominale} indica la massima potenza dissipabile a temperatura ambiente, normalmente \textit{25 °C}, senza che la sua resistenza venga alterata.
Le resistenze sono classificate in serie dalle \textbf{serie} che raggruppano dei valori commerciali, questa parte non verrà riassunta.\\
\vspace{1\baselineskip} 
Il valore nominale dei resistori e la tolleranza si può leggere grazie a un sistema di bande colorate poste sulle resistenze associate a dei valori tramite il codice colori.\\
Ci sono tre tipi di codici colori, quello a 4 bande, quello a 5 e quello a 6; tutti e tre seguono lo schema: \emph{Cifre significative, moltiplicatore, tolleranza}, infatti le prime 3 bande (prendendo per esempio i codici a 5 bande) indicano le cifre significative che indicano le prime cifre del valore della resistenza, la $4^a$ cifra indica l'ordine di grandezza o moltiplicatore $10^n$ mentre l'ultima ne indica la tolleranza, nei codici a 6 bande viene indicata dalla $6^a$ banda il coefficiente di temperatura che indica come varia la resistenza in base alla temperatura.
\vspace{\baselineskip}[H]
Una caratteristica intrinseca molto importante è la  \textbf{potenza dissipabile}, essa indica la potenza massima che il resistore è in grado di dissipare senza che si rovini ossia senza raggiungere valori troppo alti dal momento che l'energia dissipata viene trasformata in calore.\\
Per calcolarla bisogna tenere conto che se il resistore è più caldo della temperatura ambiente parte del calore viene assorbito \textit{potenza dissipata}.
\begin{center}
    La cui formula è: $P_d=\frac{T_i-T_a}{R_th}$.\\
$T_i$ = temperatura interna del componente, $T_a$ = temperatura ambiente, $R_th$=resistenza termica.
\end{center}
Ovviamente la potenza dissipabile è inversamente proporzionale alla temperatura ambiente.\\
Questa variazione è prevedibile e calcolabile grazie al coefficiente di temperatura (lettera $\alpha$) e dalla formula:
\begin{center}
    $R=R_0\cdot[1+\alpha\cdot(T-T_0)]$
\end{center}
Allo stesso modo oltre alla resistenza de viene influenzata la resistività.\\
%___________________________----------------=====================================----------------___________________________do di
\subsection{Tecnologie costruttive}
La parti che formano un resistore sono:
\begin{itemize}
    \item elemento resistivo
    \item supporto dell'elemento resistivo
    \item rivestimento di protezione
    \item reofori
\end{itemize}
I vari tipi di resistori si distinguono per come l'\textbf{elemento resistivo}, ossia la parte che viene attraversata dalla corrente, viene utilizzato.\\
Ad esempio nei \textbf{resistori a impasto} l'elemento resistivo è composto da un agglomerato di materiali e \emph{polvere di carbone}.\\
Nei \textbf{resistori a film} è costituito da uno strato di conduttore avvolto su uno strato di isolante, per aumentare la resistenza si riduce la sezione e si aumenta la resistenza dal momento che $R=\rho\cdot\frac{l}{S}$\\
Mentre nei\textbf{ resistori a filo} l'elemento resistivo è un filo metallico avvolto su materiale isolante con un "doppio avvolgimento" o \textbf{avvolgimento antinduttivo} per evitare fenomeni elettromagnetici.\\
Il materiale con cui viene fatto \textbf{supporto dell'elemento resistivo} deve avere una resistenza di isolamento elevata e proprietà meccaniche e termiche sufficienti a resistere alle temperature.\\
Rivestimento, resine o smalti.
%___________________________----------------=====================================----------------___________________________
\subsection{Tipi di resistori}
Ci sono due modi in cui i resistori si collegano ai circuiti: tramite reofori o THT \textit{Through Hole Technology} oppure saldati al circuito o SMT \textit{Sourface Mounted Technology}.
\vspace{1\baselineskip}
\\Ci sono anche dei componenti che variando la resistenza riescono a modulare un valore come:
\begin{itemize}
    \item \textbf{trimmer}\\
    I trimmer sono resistenze variabili tramite una vite che scorre sull'elemento resistivo, non sono concepiti per essere variati molto.
    \item \textbf{potenziometri}\\
    i potenziometri vengono utilizzati per variare la tensione tramite partitore di tensione.
    \item \textbf{reostati}\\
    i reostati vengono usati per regolare la corrente. $I=\frac{V_g}{R_v+R_u}$ dove $Rv$ è la resistenza del trimmer.
\end{itemize}

%___________________________----------------=====================================----------------___________________________
\section{Unità 3: Condensatori}
\subsection{Generalità}
Un condensatore è formato da due armature separato da un dielettrico e ha la finzione di immagazzinare carica elettrica per poi cederla, quando è carico alle sue due stremità ci sarà una \textit{ddp} che dipende dalla quantità di \textbf{carica} e dalla \textbf{capacità}: $V_c=\frac{Q}{C}$\\
La carica di un condensatore p definibile come "L'attitudine di immagazzinare carica elettrica".\\
Se il condensatore è alimentato da un generatore di tensione si caricherà istantaneamente mentre con una resistenza in serie si caricherà esponenzialmente in $4,5\cdot\tau$ e $\tau=Req\cdot C$.\\
La formula generale della tensione o della carica di un condensatore è:
\begin{center}
    $V_{(t)}=V_{0}(1-e)^{-t/\tau}$\\
    $Q_{(t)}=Q_{0}(1-e)^{-t/\tau}$
\end{center}

\subsection{Parametri}
I principali parametri che caratterizzano un resistore sono:
\begin{enumerate}
    \item \textbf{capacità nominale}\\
    È il valore capacitivo determinato dal produttore, si misura in farad [F] e varia in base al dielettrico e alla geometria.\\
    Per avere una grande capacità è necessario avere un dielettrico molto sottile e con molta superficie oltre a una elevata costante dielettrica poiché $C=\epsilon\cdot\frac{S}{d}$
    \item \textbf{tolleranza}\\
    Analoga alle resistenze tranne per il fatto che può essere anche asimmetrica ad esempio $ -20\%\div+50\%$\\
    \item \textbf{coefficiente di temperatura}\\
    Indica il variare della capacità in base alla temperatura calcolabile in modo analogo alle resistenze:
    \begin{center}
        $C=C_0\cdot[1+T_c\cdot(T-T_0)]$\\
        $T_c$= Temperature Coefficient
    \end{center}

    \item \textbf{tensione nominale }\\
    È la massima tensione sopportabile dal dielettrico prima di bucarsi, quindi possiamo dedurre che questa grandezza sia inversamente proporzionale alla \emph{capacità}
    \item \textbf{resistenza di isolamento}\\
    La resistenza del dielettrico, si calcola analogamente a quella dei resistori $R=\rho\cdot\frac{d}{S}$
    \item \textbf{angolo di perdita}\\
    Rappresenta la perdita di energia dato dalle resistenze che dissipano energia per l'effetto Joule, la \emph{tangente} di questo angolo rappresenta il rapporto tra potenza persa e potenza reattiva cioè il \emph{fattore di dissipazione}.
\end{enumerate}
Alcuni di questi parametri sono riportati sui condensatori più grossi o nei \textit{datasheet}. Nel primo caso vengono riportate tensione nominale, capacità e tolleranza tramite lettere o codici alfanumerici.

\subsection{Tecnologie costruttive}
Ci sono 4 tipi di condensatori:
\begin{itemize}
    \item \textbf{condensatori ceramici}\\
    Chiamati così perché appunto usano la  ceramica come dielettrico, sono a forma di disco e di tubetto e vanno da pochi pF a qualche nF.\\
    Sono divisi in 3 classi, quelli della \textit{I classe} sono \textit{molto stabili} a livello di temperatura. 
    Quelli di \textit{II classe} hanno un comportamento \textit{poco lineare} ma hanno una $\epsilon$ maggiore di quelli di classe I e posso essere stabili o instabili.\\
    Mentre i condensatori di \textit{III classe} sono caratterizzati dallo spessore estremamente \textit{sottile} del \textit{dielettrico} per ridurre al massimo le dimensioni, ma di conseguenza la tensione nominale avrà un valore molto basso.
    
    \item \textbf{condensatori a film plastici}\\
   Questi condensatori usano come dielettrico un film di \textit{plastica metallizzata} come polistirolo, teflon ecc...\\
   Vengono utilizzati a bassa frequenza, hanno un valore che va da 1 nF a  1000nF e sono molto stabili in perdite e temperatura.
    
    \item \textbf{condensatori elettrolitici}\\
    Utilizzano come dielettrico un \textbf{ossido di metallo}, utilizzano come armature il metallo (alluminio o tantalio); i condensatori in alluminio usano $AI_2O_3$ come ossido.\\
    Il dielettrico è particolarmente sottile perché viene \textit{prodotto} tramite \textit{elettrolisi} utilizzando anche un foglio di rame, l'ossido deve venire riprodotto quando è in funzione collegando una delle due armature con l'elettrolita al + altrimenti l'ossido sparisce provocando una corrente intensa e la seguente esplosione data dall'aumento di temperatura; si dice che i condensatori sono polarizzati.\\
    I condensatori al tantalio si differenziano da quelli in alluminio solo per la porosità che permette una costruzione di foma a goccia e a parallelepipedo oltre al cilindro.\\ 
    Coprono valori da 1$\mu$F a 100mF, hanno valori nominali basse e hanno una piccola resistenza di isolamento.
    
    \item \textbf{condensatori variabili}\\
    Usano come dielettrico l'aria e sono formati da lamine di alluminio disposte a libro collegate in modo alternato,di due gruppi di armature uno è fisso e l'altro mobile, consentendo di variare la capacità cambiando la superficie in comune tra due armature.
    \begin{figure}[H]
        \centering
        \includegraphics[scale=0.2]{/home/rib/varie/capscaps.jpg}
        \caption{Condensatore variabile}
        \label{varcap}
    \end{figure}
\end{itemize}
    
\section{Unità 4: Induttori}
\subsection{Generalità}
Un circuito percorso da corrente genera nello spazio un \textbf{campo magnetico} $\Phi$ questa grandezza è direttamente proporzionale 
alla corrente e dipende anche da una costante di proporzionalità chiamata \textit{induttanza}.
\begin{center}
    $\Phi=L\cdot I$
\end{center}
L'induttanza è la proprietà di un circuito di opporsi alla variazione di corrente nel tempo ha come \textit{u.d.m.} l'\emph{henry [H]}, l'induttore è il componenete che usa questa 
proprietà e al livello più semplice è una bobina.\\
A livello reale si schematizza con una resistenza in serie per raggruppare le perdite ohmice e un condensatore in parallelo per gli effetti capacitivi tra le spire.\\
\begin{figure}[H]
    \centering
        \includegraphics[scale=0.2]{/home/rib/varie/indutt.jpg}
        \caption{Schema induttore reale}
        \label{indutt}
\end{figure}
\noindent
Questa resistenza $R_s$ causa una caduta di potenziale che crea un'angolo di perdita (90°-$\delta$) da cui dipende la tensione ai capi dell'induttore.
\subsection{Parametri}
I \textbf{parametri caratteristici} di un induttore sono:
\begin{itemize}
    \item il fattore di qualità (Q);
    \item l'angolo di perdita;
    \item la frequenza di risonanza;
    \item il valore di induttanza L;
    \item la tolleranza sull'induttanza;
    \item il coefficente di temperatura dell'induttanza.
\end{itemize}
Il fattore di qualità \textbf{Q} rappresenta il rapporto tra energia immagazzinata e dissipata in un periodo \emph{t}.
\begin{center}
    \begin{math}
        Q=\omega\frac{W}{P}=2\pi\textit{f}\frac{W}{P}
    \end{math}\\
    \textit{W} è l'energia immagazzinata mentre \textit{P} è l'energia dissipata.
\end{center}
Q dipende anche dalle perdite per effetto joule, perdite dielettriche, perdite date da correnti parassite e perdite date dall'isteresi ma spoprattutto dalla frequenza.\\
L'\textbf{angolo di perdita} viene utilizzato dalle case produttrici come fattore di qualità infatti viene indicato anche con dei grafici.
La \textbf{tolleranza} è un valore che rappresenta la soglia oltre la quale l'induttore si comporta come un condensatore puro.
\begin{center}
    \textit{f}$_0=\frac{1}{2\pi \sqrt{LC}}$
\end{center}
\subsection{Tecnologie costruttive}
Gli induttori possono essere costruiti senza nucleo o con nucleo,
negli induttori \textbf{senza nucleo} l'avvolgimento viene disposto intorno a un supporto di materiale isolante.\\
Per minimizzare le perdite e aumentare la frequenza sopportabile sono stati adottati vari tipi di avvolgimento.
\begin{itemize}
    \item \textbf{solenoidale}:\\
    L'avvolgimento viene svolto con un solo strato su supporto isolante cilindrico.
    \item \textbf{a più strati} \\
    Per ottenere valori induttivi maggiori, simile al solenoide solo a più strati
    \item \textbf{a nido d'ape} \\
    Avviene translando e ruotando contemporaneamente il supporto 
    \item \textbf{toroidale}\\
    Utilizzato per campi magnetici di debole intensità, anche su più strati.
\end{itemize}
Gli induttori \textbf{con nucleo} permettono di ottenere un valore di induttanza maggiore facendo diminuire \emph{Q} per le perdite date dal ciclo di isteresi.
Il materiale migliore per i nuclei è la ferrite.
\subsubsection{Schermatura}
A volte è necessario \textbf{schermare} le bobine per non disturbare altri componenti tramite \textit{schermi protettivi}.
Nei circuiti a bassa frequenza lo schermo deve essere più spesso al diminuire della frequenza, necessita inoltre di una continuità nella struttura. Vengono utilizzati metalli ad alta permeabilità iniziale.\\
Mentre ad alta frequenza (maggiore di 50KHz) gli schermi sono realizzati con materiali ad \textbf{alta conducibilità} come alluminio, sfruttando le \emph{correnti parassite} per coninare il campo elettrico generato.

\section{Unità 4: Relè}
\subsection{Generalità}
I relè sono componenti \textit{elettromeccanici} in grado di chiudere o aprire un circuito mediante l'apertura o la chiusura di contatti elettrici.
Il cambiamento da stato avviene quando una corrente abbastanza forte percorre le \textbf{spire} della bobina avvolte sul nucleo di materiale ferromagnetico attivando il meccanismo.\\
Ci sono vari modi di cambiare stato e definiscono i tipi di relè:
\begin{itemize}
    \item \textbf{relè neutri}\\
    Se il passaggio è indipendente dal verso della grandezza della bobina.
    \item \textbf{relè polarizzati}\\
    Se il passaggio è dipendente dal verso della grandezza della bobina.
    \item \textbf{relè monostabili}\\
    Se la posizione del contatto assunta con la bobina alimentata dura solo fino a quando rimane eccitata.
    \item \textbf{relè bistabili}\\
    Se la posizione assunta con la bobina eccitata rimane anche quando cessa e per tornare al punto di partenza ha bisogno di una nuova eccitazone.
    \item \textbf{relè a tempo}\\
    Se automaticamente accadeuna variazione dello stato dei ocntatti dopo un certo intervallo di tempo da quando la bobina è stata alimentata o "spenta".
\end{itemize}
\subsection{Parametri}
I principali parametri sono:
\begin{itemize}
    \item \textbf{tension di eccitazione}\\
    Valore di tensione necessario per eccitare il relè.
    \item \textbf{tipo di corrente }\\
    Può essere continua o alternata.
    \item \textbf{rigidità dielettrica}\\
    Tensione massima applicabile senza cedimenti.
    \item \textbf{resistenza della bobina}\\
    Rapposto tra tensione di eccitazione e la corrente.
    \item \textbf{temperatura di funzionamento}\\
    Massimo valore di temperatura che la bobina può sopportare senza danneggiarsi.
    \item \textbf{durata elettrica}\\
    Numero massimo di manovre eseguibili con i contatti carichi.
    \item \textbf{durata meccanica}\\
    Numero massimo di manovre eseguibili con i contatti senza carica.
    \item \textbf{frequenza di commutazione}\\
    Numero massimo di manovre eseguibili nel periodo massimo di tempo.
    \item \textbf{tempi caratteristici}\\
    Si comprende tempo di attivazione ossia l'intervallo di tempo tra l'eccitazione della bobina e la chiusura del contatto, 
    stesso ragionamento con il tempo di rilascio solo con la condizione di riposo.\\
    In entrambi i casi si comprende il tempo di rimbalzo durante il quale gli stati variano prima di stabilizzarsi
\end{itemize}
\subsection{Tecnologie costruttive}
Si possono  distinguere due blocchi fondamentali:
\begin{itemize}
    \item \textbf{lato eccitazione}\\
    Comprende il \emph{nucleo magnetico}, \emph{avvolgimenti di eccitazione} (il numero di spire dipende dalla forza di attrazione), l'\emph{ancora} 
    \item \textbf{lato contatti}\\
    \begin{itemize}
        \item \textit{contatti di chiusura}\\
        normalmente aperti (NO)
        \item \textit{contatti di apertura}\\
        normalmente chiusi (NC)
        \item \textit{contatti di scambio}\\
        Comprendono un contatto comune tra un contatto \emph{NO} e uno \emph{NC}
    \end{itemize}

    \begin{figure}[H]
        \includegraphics[scale=0.25]{/home/rib/varie/relelele.jpg}
        \caption{Schema contatti}
        \label{contatt}
     \end{figure}
I materiali per i contatti sono oro, argento, platino, o leghe per potenze maggiori.
\end{itemize}

\end{document}